%%%%%%%%%%%%%%%%%%%%%%%%%%%%%%%%%%%%%%%%%
% Medium Length Professional CV
% LaTeX Template
% Version 3.0 (December 17, 2022)
%
% This template originates from:
% https://www.LaTeXTemplates.com
%
% Author:
% Vel (vel@latextemplates.com)
%
% Original author:
% Trey Hunner (http://www.treyhunner.com/)
%
% License:
% CC BY-NC-SA 4.0 (https://creativecommons.org/licenses/by-nc-sa/4.0/)
%
%%%%%%%%%%%%%%%%%%%%%%%%%%%%%%%%%%%%%%%%%

%----------------------------------------------------------------------------------------
%	PACKAGES AND OTHER DOCUMENT CONFIGURATIONS
%----------------------------------------------------------------------------------------

\documentclass[
	%a4paper, % Uncomment for A4 paper size (default is US letter)
	11pt, % Default font size, can use 10pt, 11pt or 12pt
]{resume} % Use the resume class

% \usepackage{ebgaramond} % Use the EB Garamond font

\addbibresource{publications/journal_articles.bib}
\addbibresource{publications/conference_papers.bib}
\addbibresource{publications/presented_abstracts.bib}

%------------------------------------------------

\name{Micah Oevermann} % Your name to appear at the top

% You can use the \address command up to 3 times for 3 different addresses or pieces of contact information
% Any new lines (\\) you use in the \address commands will be converted to symbols, so each address will appear as a single line.

\address{College Station, Texas} % Main address

% \address{123 Pleasant Lane \\ City, State 12345} % A secondary address (optional)

\address{mjooevermann@gmail.com} % Contact information

%----------------------------------------------------------------------------------------

\begin{document}

%----------------------------------------------------------------------------------------
%	EDUCATION SECTION
%----------------------------------------------------------------------------------------
\begin{rSection}{Education}

	\textbf{PhD in Mechanical Engineering} \hfill \textit{December 2025} \\          
    Texas A\&M University, College Station\\ 
	Advisor: Dr. Robert Ambrose \smallskip\\
	% Member of Upsilon Pi Epsilon \\
	\textbf{B.S. in Mechanical Engineering} \hfill \textit{December 2021} \\ 
	 Texas A\&M University, College Station\\
	
\end{rSection}

%----------------------------------------------------------------------------------------
%	WORK EXPERIENCE SECTION
%----------------------------------------------------------------------------------------
\begin{rSection}{Positions Held}
    \begin{rSubsectionWorkList}{Robotics Automation and Design Lab}{Janurary 2022 - Present}{Graduate Research Assistant}{College Station, TX}
    \end{rSubsectionWorkList}
    \begin{rSubsectionWorkList}{BakerRisk Engineering Consultants}{August 2020 - December 2020}{Student Co-op, Blast Testing Group}{San Antonio, TX}
    \end{rSubsectionWorkList}
    \begin{rSubsectionWorkList}{Biomechanical Environments Laboratory}{January 2019 - May 2019}{Undergrad Research Assistant}{College Station, TX}
    \end{rSubsectionWorkList}
\end{rSection}

\begin{rSection}{Professional Projects}

	\begin{rSubsection}{RoboBall II}{RAD Lab}{}{}
		\item Donec et mollis dolor. Praesent et diam eget libero Adobe Coldfusion egestas mattis sit amet vitae augue.
		\item Nam tincidunt congue enim, ut porta lorem Microsoft SQL lacinia consectetur.
		\item Donec ut libero sed arcu vehicula ultricies a non tortor. Lorem ipsum dolor sit amet, consectetur adipiscing elit.
		\item Pellentesque auctor nisi id magna consequat JavaScript sagittis.
		\item Aliquam at massa ipsum. Quisque bash bibendum purus convallis nulla ultrices ultricies.
	\end{rSubsection}

%------------------------------------------------

	\begin{rSubsection}{Deflagration Load Generator Testing}{BakerRisk}{}{}
		\item Testing support for Deflagration Load, Vapor Cloud Explosion, and Shock Tube tests
            \item Set up atmospheric pressure gauge array in testing zone 
            \item Debug issues in the field hardware, electrical, and networking support systems
		\item Trusted to operate the largest shock tube in the world outside of government agencies
		\item Manufactured a custom interchangeable testing mount for glass windows
		\item Successfully tested 25 shock-resistant windows for an outside client
		\item Prioritized safety with no major injuries while working around broken glass
	\end{rSubsection}

%------------------------------------------------

	\begin{rSubsection}{Bi-axial Tissue Tensile Characterization}{Biomechanical Environments}{}{}
		\item Applied concepts of linear elastic theory in the development of a biaxial tissue testing platform
		\item Prepared and marked organic tissue samples for use in testing
		\item Implemented the use of a novel fish hook – line technique to reduce clamp stresses
		\item Presented final design on a poster in a public research symposium
	\end{rSubsection}

\end{rSection}

%----------------------------------------------------------------------------------------
%	Personal Projects
%----------------------------------------------------------------------------------------
\begin{rSection}{Personal Projects}

    \begin{rSubsection}{Auto-Update CV}{2025}{}{}
        \item This CV auto-updates the publications list by scraping my Google Scholar profile with a Python script, then compiling the updated \LaTeX \ bib files with Github Actions 
    \end{rSubsection}

    \begin{rSubsection}{Reddit 6-DOF Arm}{2022}{}{}
        \item This CV auto-updates the publications list by scraping my Google Scholar profile with a Python script, then compiling the updated \LaTeX \ bib files with Github Actions 
    \end{rSubsection}
    
\end{rSection}
%----------------------------------------------------------------------------------------
%	TECHNICAL STRENGTHS SECTION
%----------------------------------------------------------------------------------------

\begin{rSection}{Technical Skills}

	\begin{tabular}{@{} >{\bfseries}l @{\hspace{6ex}} l @{}}
		Computer Science & Python, Cpp, Docker, ROS2, LCM, CAN \\
		Electrical Engineering & Soldering, Cable Harnessing,  \\
            Mechanical Engineering & Solidworks, Milling, 3D Printing, Design for Assembly
	\end{tabular}

\end{rSection}

%----------------------------------------------------------------------------------------
%	Publications List
%----------------------------------------------------------------------------------------
\pagebreak
\begin{rSection}{Publications}

\section*{Journal Articles}
    
    \nocite{*} % cite all in the bib files
    
    \printbibliography[
    heading=none,
    type=article
    ]

\section*{Peer Reviewed Conference Papers}
    
    \nocite{*} % cite all in the bib files
    
    \printbibliography[
    heading=none,
    type=inproceedings
    ]

\section*{Presented Abstracts}
    
    \nocite{*} % cite all in the bib files
    
    \printbibliography[
    heading=none,
    type=misc
    ]
	

\end{rSection}

%----------------------------------------------------------------------------------------
%	EXAMPLE SECTION
%----------------------------------------------------------------------------------------

%\begin{rSection}{Section Name}

	%Section content\ldots

%\end{rSection}

%----------------------------------------------------------------------------------------

\end{document}
